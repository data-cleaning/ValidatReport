\section{Validation reports}
\label{sect:basicreports}
Recall that a \emph{validation} is a tuple $(e,d,f,v)$ and an
\emph{aggregation} is a tuple $(e,d,f,a)$. Here ($v\in\{0,1,\na{}\})$ is a
validatin result and $a$ is an aggregate value.  In both cases, $e$ refers to
the event that where the expression $f$ was evaluated to create the result
($v$, or $a$) and $d$ refers to the data evolved in evaluating the expression
as well as the data related to the interpretation of the result. 

The tuples are constructed to make  the values $a$ and $v$ identifiable
(Demand~\ref{dem:identify}).  The following recursive construction defines
validation reports that are combinable (Demand~\ref{dem:combine}) and
aggregable (Demand~\ref{dem:aggregate}).
%
\begin{definition}[Validation report]\leavevmode
\begin{enumerate}[topsep=0pt,itemsep=0pt]
\item The empty set $\{\}$ is a validation report.
\item If $(e,r,d,v)$ is a  validation then $\{(e,d,r,v)\}$ is a validation report.
\end{enumerate}
Note that $\{(e,d,r,v)\}$ is a trivial DAG, with a single node and no edges.
\begin{enumerate}
\setcounter{enumi}{2}
\item If $V$ and $W$ are combinable in the sense of
Definition~\ref{def:compatibledag}, then $V\cup W$ is also a validation report.
\item If $V$ is a validation report, $f$ is an expression, and  $S$ is a subset of $V$  
such that $f$ can be evaluated with $S$. Then 
\begin{align*}
V\cup \{(e_{fS}, d_{fS}, f, f(S))\},
\end{align*}
is also a validation report. Here, $e_{fS}$ identifies the event that created
the aggregate $f(S)$ and $d_{fS}$ identifies $S$ and the data to which the
result $f(S)$ pertains.
\end{enumerate}
\label{def:basicvalidationreport}
\end{definition}
%
The first step is a formality, allowing for the edge case of empty reports.  By
applying the second and third step repeatedly, a report can be populated with
identifiable validation results (validations). Observe that as long as we are
only adding validations, the validation reports are trivially combinable since
it can be interpreted as an edgeless graph (see under
definition~\ref{def:compatibledag}). In step four, an identifiable aggregate
(aggregation) is constructed that is compatible with the validation report to
which it is added. Remember that an aggregation object also stores the edges to
the nodes that were used to construct it so this definition indeed constructs a
directed acyclic graph.

Now that we have a conceptual definition of a validation report, that satisfies
all three demands, we can move forward and define the identifying pieces of
information to be stored in the  validations and aggregations on a logical
level.


%%%%%%%%%%%%%%%%%%%%%%%%%%%%%%%%%%%%%%%%%%%%%%%%%%%%%%%%%%%%%%%%%%%%%%%%%%%%%%%
\subsection{Logical validation report structure}
\label{sect:basicreportstructure}
In the following subsections the information that needs to be stored in
validation or aggregation tuples is described explicitly. The descriptions
are formatted in a set of tables, each with the following structure.
%
\begin{enumerate}
\item Item: the name of the information item.
\item Format: logical format of the data in the item. Allowed formats are: \code{string},
\code{numeric}, \code{enum} (with categories defined in the description
column), \code{datetime} and \code{-}. The latter indicates that the format is
free, including the possibility to include user-defined objects. A type
may be followed by brackets \code{[]} to indicate an array.
\item Description: a short description of the item. More detailed descriptions
might follow after the table.
\item Example: an example.
\end{enumerate}
%
We distinguish between information which is mandatory and information that is
recommended. This is indicated in the caption of each table. Some information
items may be extended with user-defined information. Whether this is the case
is indicated at the top of each table.


%%%%%%%%%%%%%%%%%%%%%%%%%%%%%%%%%%%%%%%%%%%%%%%%%%%%%%%%%%%%%%%%%%%%%%%%%%%%%%%
\subsubsection{Identification of an expression evaluation event.}
\label{sect:idevent}
Both validation results and aggregates are created by an event $e$ that
evaluates an expression. 

\begin{spec}{
Mandatory identification of a expression evaluation event $e$
}{}
time             & \code{datetime} & Time marking the completion of a validation event. & \code{20170212 10:15:30+0100}\\
actor         & \code{string}   & Software that or person who created the result. & \code{R package validate version 0.1.7}\\
\end{spec}

The data validation business architecture \citep{ess2017} defines a
service-oriented infrastructure for data validation. In the cases where a
client-server model is applied (the server executing validation and sending
reports), the following extra information is recommended.

\begin{spec}{Recommended information on a physical validation event $e$}{}
agent   & \multicolumn{1}{c}{-} & Actor (person, institute, dpt, $\ldots$) responsible for executing the validation event & dpt. of data validation, Eurostat\\
trigger & \multicolumn{1}{c}{-} & Actor (person, institute, dpt, $\ldots$) responsible for triggering the event  & John Statistician, Statistics Netherlands\\
\end{spec}

%%%%%%%%%%%%%%%%%%%%%%%%%%%%%%%%%%%%%%%%%%%%%%%%%%%%%%%%%%%%%%%%%%%%%%%%%%%%%%%
\subsubsection{Identification of an expression}
\label{sect:idrule}
%
\begin{spec}{Mandatory identification of an expression $f$}{}
language      & string   & Language and version in which a expression is written & R/validate version 0.1.7\\
expression    & string   & Expression defining the rule or aggregate.           & \code{age >= 0}\\
severity$^\dagger$      & enum     & \code{`error'}, \code{`warning'},
                           or \code{`information'}                & \code{`error'}\\ 
\end{spec}

The ESS validation business architecture also allows for certain up- or
downgrades of the severity status for individual cases. Furthermore, it is good
practice to explain the purpose of complicated expressions in a human-readable
description.

\begin{spec}{Recommended values for identification of a validation rule}{}
description   & \code{string} & human-readable description of the rule. & 
Nonnegativity for age.\\
status$^\dagger$        & \code{string} & Whether the severity level was up- or downgraded for 
the current report. & Lowered from \code{error}.\\
\end{spec}

Of course the `status' field can also be used to add an explanation on why a
status was changed.

%%%%%%%%%%%%%%%%%%%%%%%%%%%%%%%%%%%%%%%%%%%%%%%%%%%%%%%%%%%%%%%%%%%%%%%%%%%%%%%
\subsubsection{Identification of data}
\label{sect:iddata}
The validation report identifies two data sets for each reported validation
result or aggregate: the set of datapoints that was involved in evaluating the
expression and the set of datapoints related to the interpretation of the
result. The dataset that is used to evaluate an expression will be referred to
as the \emph{source} data while the dataset that is related to the
interpretation of the result will be referred to as the \emph{target} data. In
many cases these will coincide but  see Equation~\ref{eq:hbfun} on
Page~\pageref{eq:hbfun} for a counterexample.


\begin{spec}{Mandatory identification of validated data.}{}
source    & string[] 
  & A key or set of keys identifying the data used in evaluating the expression.
  &  $\{$(`Dutch inhabitants', `EU-SILC2016, `Richard Respondent', `Income')$\}$\\
target    & string[] 
  & A key or set of keys identifying the data targeted by the expression.
  &  $\{$(`Dutch inhabitants', `EU-SILC2016, `Richard Respondent', `Income')$\}$\\
\hline
\multicolumn{4}{|l|}{\textbf{Convention:} if the `target' field is empty,
it is assumed equal to `source'.}\\
\end{spec}


Since a set of keys that identify a dataset is hard to interpret by humans, we
add the following recommendation.

\begin{minipage}{\textwidth}
\begin{center}
\captionof{table}{Recommended values for identification of validated data.}
\begin{tabular}{|lp{0.1\textwidth}p{0.34\textwidth}p{0.30\textwidth}|}
\hline
\textbf{Item} & \textbf{Format} & \textbf{Description} &\textbf{Example}\\
\hline
description   & \code{string} & human-readable description of the data. & 
Income of a single citizen.\\
\hline
\end{tabular}
\end{center}
\end{minipage}

Below, we sketch two possible ways on how the data identification could be
implemented.

\paragraph{1. Full specification of data.} The methodology handbook on validation
prescribes a generic model to identify a single datapoint
\citep[Chapter~5]{zio2015methodology}. In short, one identifies the value of a
data point by fixing
\begin{itemize}
\item the population $U$ to which it pertains;
\item the event $\tau$ that lead to its observation;
\item the population unit $u$ from which a property was observed, and
\item the attribute $X$ that was measured.
\end{itemize}
Here, the term `population' should be interpreted rather generally. It may be
the human population of a country or region, but it can also be a population of
companies, countries, events, emails, and so on. Similarly, the event that lead
to an observation can be the receiving of transmitted data from an institute,
or it may be a data collection event based on a survey.  In the handbook, a
data point is defined as a value (from some domain) paired with a tuple
$(U,\tau,u,X)$ that identifies it.


When the set of keys consists of a set of $(U,\tau,u,X)$-tuples as defined in
the methodological handbook on validation, the report will identify data
involved in validation completely free of any context involving the sender, the
process, institutes involved and so on. 

\paragraph{2. Extra standardization.} The key sets can quickly inflate the size
of a validation report. Since the format is left open (we only specify keys to
be an (array of) strings), it is possible to apply a more practical format at
the price of extra standardizing agreements between sender and receiver of the
report. Let us illustrate this by sketching a validation procedure in a service-client
based infrastructure. We call the client `Alice' and the server `Bob'.
\begin{enumerate}[noitemsep]
\item Alice sends data, consisting of $n$ records and a validation rule to `Bob'.
She also sends a unique string $s$ (for example a hash key) that she has connected
with this particular dataset in her administration. The rule she sends is such that
it must be evaluated on every record.
\item Upon receiving Alice's message, Bob evaluates the rule on each of the $n$
records. He sets each \emph{source} field equal to the string $s.k$, where $k$
is the key that uniquely identifies the record under scrutiny. The \emph{target}
fields are left empty.
\item Bob completes the validation report and sends it to Alice.
\end{enumerate}

The trade-off in the above procedure is that the validation report can only be 
understood by the sender and receiver, but not by a third party who is unaware
of the meaning of the identifying keys $s$ and $k$.


\subsubsection{Result values}
\label{sect:valres}

\begin{spec}{
Mandatory format for the validation result $v$ or aggregation result $a$}{not }
value$^\dagger$  & \code{enum}   & $1$, $0$, or \na{}    &$1$\\
value         & \code{string} & evaluation result     &\code{"7"}\\
\end{spec}

