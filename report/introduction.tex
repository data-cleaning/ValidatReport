\section{Introduction}
\label{sect:introduction}

Data validation is at the core of every production chain in official statistics.
Whether it is the input received from a survey, a bunch of data records transferred from an administrative source or the result from some internet scraping,
data must be checked against its expectations to process it and turn it into reliable statistics.
For a more detailed explanation of data validation in general and the principles we identified in validation, we refer to the 
handbook of data validation from the ESSnet Foundation, by Di Zio et al. (2015) and the validation principles written down by the Eurostat validation task force, see ESS (2017).

Standardisation is important in official statistics and this also applies to validation processes.
Especially in the case of cross-organisation validation, where both a data producer and a data receiver check the same data against some
commonly agreed validation rules, standardisation is crucial to prevent different interpretations of the results.
The more harmonised validation reports are, the better the understanding across organisations is.
This has been recognized by the ESSnet on Validation project (Validat Integration).
A Work Package (WP2) was defined to attack this issue.
This report is one of the deliverables of Work Packages 2.
It contains the result of the work carried out by various project partners and focusses on the standardisation of the output of a validation process: the \emph{validation report}.

Obviously we are interested in the question what information can and should be included in a validation report and the optimal way to express this information.
In this document we develop a \emph{generic} structure for expressing validation results.
We have the ambition to develop a validation report structure that can be used in \emph{any validation task} in \emph{any organisation} in any \emph{statistical domain}. 
To make it applicable in machine to machine communication contexts as well as in human contexts we design a machine-readable as well as a human readable format.

The approach we have taken is a combination of a top-down apprach and a botom-up approach.
In the bottom-up approach a number of example validation reports from member states and Eurostat were collected and studied to identify common elements.
This resulted in a long-list of validation report elements categorized into several classes such as rule metadata, process metadata, aggregates etc.
This led to some thinking about the basic concepts that are used in validation reports across the ESS.
We used the results from the bottom-up approach to develop a more formal top-down approach expressed in this report.
Step by step we built a validation report structure that is generic enough to be used in a wide range of validation contexts in many institutes,
expressive enough to support the most recognized validation report elements recognized from the bottom-up approach and
flexible enough to be adapted in a regional validation context while containing the standardized elements from the generic format.
Validation tools or other statistical tools producing validation reports as a side product should be able to use this structure for their validation output.

In this deliverable we address the following aspects:
In chapter 2 we ask ourselves what we should expect from a generic validation report structure.
After some elaboration on the variety in richness of a validation report, we define a number of generic demands.
In chapter 3 we define some core elements to be used in the definition of the validation report structure such as a validation result and a validation event.
In chapter 4 we formally define a validation report and propose a JSON-concept scheme for a basic machine-readable validation report.
This basic scheme leaves out aggregation facilities, which complicates the design considerably.
We describe the possibilities to add aggregation facilities to the report structure in chapter 5 and propose a JSON-concpet scheme for such message as well.
Chapter 6 shows some examples of validation reports, expressed in the schemes derived in chapters 4 and 5.
chapter 7 explains how a machine readable format can be transformed into a human readable format.
Chapter 8 presents a brief conclusion.

