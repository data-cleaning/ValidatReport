\section{Basic validation reports}
\label{sect:basicreports}
The concepts of a validation result and a validation event now allow us to give
a recursive definition of a basic validation report. These reports are aimed to
satisfy the demands for results to be identifiable and reports to be
composable.
%
\begin{definition}[Basic validation report]\leavevmode
\begin{enumerate}[topsep=0pt,itemsep=0pt]
\item The empty set $\{\}$ is a basic validation report.
\item If $(e,r,d,v)$ is a confrontation then $\{(e,d,r,v)\}$ is a basic
validation report.
\item If $V$ and $W$ are basic validation reports, then the set union $V\cup W$
is also a validation report.
\end{enumerate}
\label{def:basicvalidationreport}
\end{definition}
%
The first three elements of each confrontation make sure that the associated
result becomes identifiable (Demand~\ref{dem:identify}). The set structure and
inclusion of set union in the definition guarantees combinability of the
reports (Demand~\ref{dem:combine}).

It follows immediately that any subset of a basic validation report is a basic
validation report. Also we inherit from set theory that for basic validation
reports $V$ and $W$, the set difference $V-W$ is also a validation report as
well as the set intersection $V\cap W$.

Now that we have a conceptual definition of a basic validation report, we can
move forward and define the identifying pieces of information to be stored in
the confrontations on a logical level.


%%%%%%%%%%%%%%%%%%%%%%%%%%%%%%%%%%%%%%%%%%%%%%%%%%%%%%%%%%%%%%%%%%%%%%%%%%%%%%%
\subsection{Basic validation report structure}
\label{sect:basicreportstructure}
In the following we present the information to be stored in a basic validation
report in a number of tables. The information is chosen so validation results
are identifiable. In these tables, each row represents an information item to
be included. Each row has four colums with the following information:
%
\begin{enumerate}
\item Item: the name of the information item.
\item Format: logical format of the data in the item. Allowed formats are: \code{string},
\code{numeric}, \code{enum} (with categories defined in the description
column), \code{datetime} and \code{-}. The latter indicates that the format is
free, including the possibility to include user-defined objects. A type
may be followed by to brackets \code{[]} to indicate an array.
\item Description: a short description of the item. More detailed descriptions
might follow after the table.
\item Example: an example.
\end{enumerate}
%
We distinguish between information which is mandatory and information that
is recommended. This is indicated in the caption of each table. Some
information items may be extended with user-defined information. Whether
this is the case is indicated at the bottom of each table.


%%%%%%%%%%%%%%%%%%%%%%%%%%%%%%%%%%%%%%%%%%%%%%%%%%%%%%%%%%%%%%%%%%%%%%%%%%%%%%%
\subsubsection{Identification of a physical validation event.}
\label{sect:idevent}
Recall that a \emph{confrontation} is defined as a tuple $(e,r,d,v)$ where $r$
and $d$ identify the data and rules confronted, $e$ identifies the actual
physical event that generated the result and $v$ is the resulting value.
(Definition~\ref{def:confrontation}). This section identifies the information
components that make up $t$.

\begin{minipage}{\textwidth}
\begin{center}
\captionof{table}{Mandatory identification of a physical validation event.}
\label{tab:idve}
\begin{tabular}{|lp{15mm}p{0.34\textwidth}p{0.34\textwidth}|}
\hline
\textbf{Item} & \textbf{Format} & \textbf{Description} &\textbf{Example}\\
\hline
time          & \code{datetime} & Time marking the completion of a validation event. & \code{20170212T101530+0100}\\
actor         & \code{string}   & Software that or person who created the validation result. & \code{R package validate version 0.1.7}\\
\hline
\multicolumn{4}{|l|}{The number of slots defining a validation event is \textbf{extendable}.
}\\
\hline
\end{tabular}
\end{center}
\end{minipage}

\begin{center}
\captionof{table}{Recommended information on a physical validation event.}
\begin{tabular}{|lp{15mm}p{0.34\textwidth}p{0.34\textwidth}|}
\hline
\textbf{Item} & \textbf{Format} & \textbf{Description} &\textbf{Example}\\
\hline
agent   & \multicolumn{1}{c}{-} & Actor (person, institute, dpt) responsible for executing the validation event & dpt. of data validation, Eurostat\\
trigger & \multicolumn{1}{c}{-} & Actor (person, institute, dpt) responsible for triggering the event  & John Statistician, Statistics Netherlands\\
\hline
\end{tabular}
\label{tab:recommendedevent}.
\end{center}
The recommended slots of Table~\ref{tab:recommendedevent} are useful, especially when validation is executed as a (remote) service.


%%%%%%%%%%%%%%%%%%%%%%%%%%%%%%%%%%%%%%%%%%%%%%%%%%%%%%%%%%%%%%%%%%%%%%%%%%%%%%%
\subsubsection{Identification of a validation rule}
\label{sect:idrule}
%
\begin{center}
\captionof{table}{Mandatory identification of a validation rule.}
\begin{tabular}{|lp{15mm}p{0.34\textwidth}p{0.34\textwidth}|}
\hline
\textbf{Item} & \textbf{Format} & \textbf{Description} &\textbf{Example}\\
language      & string   & Language and version in which a rule is expressed & R/validate version 0.1.7\\
expression    & string   & Expression defining the rule           & \code{age >= 0}\\
severity      & enum     & \code{`error'}, \code{`warning'},
                           or \code{`information'}                & \code{`error'}\\ 

\hline
\multicolumn{4}{|l|}{The number of slots defining a validation rule is \textbf{extendable}.
}\\
\hline
\end{tabular}
\end{center}

\begin{center}
\captionof{table}{Recommended values for identification of a validation rule.}
\begin{tabular}{|lp{0.1\textwidth}p{0.34\textwidth}p{0.30\textwidth}|}
\hline
\textbf{Item} & \textbf{Format} & \textbf{Description} &\textbf{Example}\\
\hline
description   & \code{string} & human-readable description of the rule. & 
Nonnegativity for age.\\
\hline
\end{tabular}
\end{center}


%%%%%%%%%%%%%%%%%%%%%%%%%%%%%%%%%%%%%%%%%%%%%%%%%%%%%%%%%%%%%%%%%%%%%%%%%%%%%%%
\subsubsection{Identification of the validated data}
\label{sect:iddata}
With the validated data, we mean the set of datapoints that was used to
evaluate the current validation rule, resulting in the reported validation
result. Since rules are are defined independent of a data set instance, this
identification can only take place during the physical validation event. To be
clear, consider the following example. Given the rule \code{turnover >= 0}.
Such a rule is implicitly understood to be checked for each turnover value
separately. Hence, for a dataset with $n$ records, $n$ validation events will
take place, each resulting in a single validation value. The data to be
identified with each result is a single data point (i.e. a single turnover
value). Now consider the rule that states \code{mean(profit) > 0}, evaluated on
the same dataset. Each profit value in the dataset is involved in the
evaluation of this rule. We therefore conclude that in general, we need to
identify a set of data points to make a validation result identifiable.


The methodology handbook on validation prescribes a generic model to 
identify a single datapoint \citep[Chapter~5]{zio2015methodology}. In short,
one identifies the value of a data point by fixing
\begin{itemize}
\item the population $U$ to which it pertains;
\item the event $\tau$ that lead to its observation;
\item the population unit $u$ from which a property was observed, and
\item the attribute $X$ that was measured.
\end{itemize}
Here, the term `population' should be interpreted rather generally. It may be
the human population of a country or region, but it can also be a population of
companies, countries, events, emails, and so on. Similarly, the event that lead
to an observation can be the receiving of transmitted data from an institute,
or it may be a data collection event based on a survey.  In the handbook, a
data point is defined as a value (from some domain) paired with a tuple
$(U,\tau,u,X)$ that identifies it.




\begin{center}
\captionof{table}{Mandatory identification of validated data.}
\begin{tabular}{|lp{15mm}p{0.34\textwidth}p{0.34\textwidth}|}
\hline
\textbf{Item} & \textbf{Format} & \textbf{Description} &\textbf{Example}\\
data    &\multicolumn{1}{c}{-} & A set of key(s) 
identifying the data used in evaluating the validation rule.
&  $\{$(`Dutch inhabitants', `EU-SILC2016, `Richard Respondent', `Income')$\}$\\
\hline
\multicolumn{4}{|l|}{The number of slots defining a validation rule is \textbf{extendable}.
}\\
\hline
\end{tabular}
\end{center}


\begin{minipage}{\textwidth}
\begin{center}
\captionof{table}{Recommended values for identification of validated data.}
\begin{tabular}{|lp{0.1\textwidth}p{0.34\textwidth}p{0.30\textwidth}|}
\hline
\textbf{Item} & \textbf{Format} & \textbf{Description} &\textbf{Example}\\
\hline
description   & \code{string} & human-readable description of the data. & 
Income of a single citizen.\\
\hline
\end{tabular}
\end{center}
\end{minipage}

When the set of keys consists of a set of $(U,\tau,u,X)$-tuples as defined in
the methodological handbook on validation, the report will identify data
involved in validation completely free of any context involving the sender, the
process, institutes involved and so on. 

The format is left open since the denotation of keys typically depends on
local storage schemes and formats. Also, the open format allows for reduction
of storage format by referring to a well defined data set and stating a query
statement that would reproduce the exact data points used in the validation
event. Such a format does require extra standardization between sender and
receiver of the report.


\subsubsection{Validation result}
\label{sect:valres}
A validation result can have three different values and is hence conceived
logically as a enumeration type.
\begin{center}
\captionof{table}{Mandatory format for the validation result $v$.}
\begin{tabular}{|lp{15mm}p{0.34\textwidth}p{0.34\textwidth}|}
\hline
\textbf{Item} & \textbf{Format} & \textbf{Description} &\textbf{Example}\\
value  & \code{enum} & $1$, $0$, or \na{}    &$1$\\
\hline
\multicolumn{4}{|l|}{The number of slots defining a validation result \textbf{not extendable}.
}\\
\hline
\end{tabular}
\end{center}


