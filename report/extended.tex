\section{Extended validation reports}
\label{sect:extended}
Conceptually, an extended validation report is a direct acyclic graph (DAG) of
which the nodes represent validation results or aggregates thereof and whose
edges convey the nodes used in aggregation. There are several ways to represent
a DAG or graphs in general: one may store a list of vertices and edges, or in
the case of a DAG one may store the outgoing edges within each node.  For the sake
of consistency with the definition of a confrontation, we choose the latter
approach. Hence the following definition.
%
\begin{definition}[aggregation]
\label{def:aggregation}
An \emph{aggregation} is a triple $(f,s,x)$ where $f$ identifies the function
used for aggregation, $s$ the set of nodes used for aggregation and $x=f(s)$ is
the aggregate.
\end{definition}
Just like Definition~\ref{def:confrontation}, this is not a definition in a
precise mathematical sense. Rather it is something that can be tested in a
particular practical software/data environment. 

Also observe that we have not specified a codomain (range) for the aggregating
function $f$. We keep the definition general by design since aggregating
functions can have several relevant output types, some examples:
\begin{itemize}
\item the number of events resulting int \waar{} (numeric);
\item the rule that is violated most often (a string identifying the rule);
\item whether all events resulted in \waar{} (validation result).
\end{itemize}
The only thing that is important for a aggregating function is that it results
in a single value, in the sense that the result pertains to the whole subset
$s$ on which $f$ acts.

We are now ready to define the extended validation report.
%
\begin{definition}[Extended validation report]\leavevmode
\label{def:extvalrep}
\begin{enumerate}
\item The empty set $\{\}$ is an extended validation report.
\item If $B$ is a basic validation report then $B$ is an extended validation report.
\item Given an extended validation report $V$, a subset $s\subseteq V$ and a
partial function $f$ on $2^V$ such that $s$ is in the domain of
$f$.  Then  $V\cup \{(f,s,f(s))\}$ is also an extended validation report. 
\item If $V$ and $W$ are extended validation reports that are compatible
in the sense of Definition~\ref{def:compatibledag}, then $V\cup W$ is
also an extended validation report.
\end{enumerate}
\end{definition}
Here, the first step is a formality, allowing one to have empty extended
validation reports. In the second step, also a formality, basic validation
reports are defined as a subtype of extended validation reports.  In step
three, a new aggregate is added.  The last step defines the combination rule,
which is complicated by the fact one should check whether two reports are
compatible. That is, one must ensure that no cycles are introduced in the DAG
represented by the extended validation report.



%%%%%%%%%%%%%%%%%%%%%%%%%%%%%%%%%%%%%%%%%%%%%%%%%%%%%%%%%%%%%%%%%%%%%%%%%%%%%%%
\subsection{Extended validation report structure}
\label{sect:extendedreport}
In the following subsections we define the logical information values to be
stored in an extended validation report. Since the definition of extended
validation reports includes basic validation reports
(Definition~\ref{def:basicvalidationreport}), we will focus on the elements
that are new, namely the \code{aggregation} concept of
Definition~\ref{def:aggregation}.  Recall that an aggregate value is determined
uniquely by identifying the set of nodes used for the agregate and the
aggregation method. Below we provide mandatory and recommended information
elements that for these concepts.


The format of the Tables defining information elements is defined at the
beginning of Section~\ref{sect:basicreportstructure}.


%%%%%%%%%%%%%%%%%%%%%%%%%%%%%%%%%%%%%%%%%%%%%%%%%%%%%%%%%%%%%%%%%%%%%%%%%%%%%%%
\subsubsection{Identification of aggregators}
\label{sect:aggregators}
An aggregator should be defined in a formal language so there can be no doubt
on its interpretation.
%
\begin{center}
\captionof{table}{Mandatory values for identification of an aggregator.}
\begin{tabular}{|lp{0.1\textwidth}p{0.34\textwidth}p{0.30\textwidth}|}
\hline
\textbf{Item} & \textbf{Format} & \textbf{Description} &\textbf{Example}\\
\hline
language   & \code{string} & Language and version in which the aggregator is expressed. & \code{R version 3.4.0}\\
expression & \code{string} & Expression used for aggregation & \code{mean(x, na.rm=TRUE)}\\
\hline
\multicolumn{4}{|l|}{The number of slots for an aggregation is \textbf{extendable}.
}\\
\hline
\end{tabular}
\end{center}

\begin{center}
\captionof{table}{Recommended values for identification of an aggregator.}
\begin{tabular}{|lp{0.1\textwidth}p{0.34\textwidth}p{0.30\textwidth}|}
\hline
\textbf{Item} & \textbf{Format} & \textbf{Description} &\textbf{Example}\\
\hline
description   & \code{string} & human-readable description of the aggregation procedure. & 
Mean, ignoring missing values.\\
\hline
\end{tabular}
\end{center}

\subsubsection{Identification of nodesets}
\label{sect:nodesets}
The nodesets represent the confrontations and/or other aggregates used to
compute an aggregate.

\begin{center}
\captionof{table}{Mandatory identification of a set of nodes.}
\begin{tabular}{|lp{0.1\textwidth}p{0.34\textwidth}p{0.36\textwidth}|}
\hline
\textbf{Item} & \textbf{Format} & \textbf{Description} &\textbf{Example}\\
\hline
nodes  & \code{string[]} & List of keys referring to the nodes used in aggregation & 
\code{["0xc0ffee"}, \code{"0xbeefed"]}\\
\hline
\multicolumn{4}{|l|}{The number of slots for an nodeset is \textbf{extendable}.
}\\
\hline
\end{tabular}
\end{center}

\begin{center}
\captionof{table}{Recommended values for identification of a nodeset.}
\begin{tabular}{|lp{0.1\textwidth}p{0.34\textwidth}p{0.30\textwidth}|}
\hline
\textbf{Item} & \textbf{Format} & \textbf{Description} &\textbf{Example}\\
\hline
description   & \code{string} & human-readable description of the nodeset. & 
fraction of events yielding \waar{}.\\
\hline
\end{tabular}
\end{center}


%%%%%%%%%%%%%%%%%%%%%%%%%%%%%%%%%%%%%%%%%%%%%%%%%%%%%%%%%%%%%%%%%%%%%%%%%%%%%%%
\subsubsection{Aggregation values}  
\begin{center}
\captionof{table}{The value of aggregation.}
\begin{tabular}{|lp{0.1\textwidth}p{0.34\textwidth}p{0.36\textwidth}|}
\hline
\textbf{Item} & \textbf{Format} & \textbf{Description} &\textbf{Example}\\
\hline
value  & \code{string} & The value computed during aggregation & 
\code{"3.141592653590"}\\
\hline
\multicolumn{4}{|l|}{The number of slots for an nodeset is \textbf{extendable}.
}\\
\hline
\end{tabular}
\end{center}

We choose to represent the aggregated value as a string, since the outcome of an
aggregate is not necessarily numerical. Aggregates may also answer both numerical 
questions such `how often is a certain rule violated?' but also qualitative questions
such as `which rule is violated most often?'.

