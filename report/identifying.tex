%%%%%%%%%%%%%%%%%%%%%%%%%%%%%%%%%%%%%%%%%%%%%%%%%%%%%%%%%%%%%%%%%%%%%%%%%%%%%%%
\section{Identifying validation results}
\label{sect:identifying}
To get an idea of the (meta)data necessary to identify a validation result,
consider the data, validation rules and validation results shown in
Table~\ref{tab:example1}. When a dataset is confronted with a validation rule,
there are three possible outcomes. A rule can be satisfied, yielding $1$, a
rule can be failed, yielding $0$, or a rule cannot be evaluated because
of missing data, yielding \na{}.
%
\begin{table}
\centering
\caption{Example data, validation rules, and results for six validation 
events.}
\begin{tabular}{rccccb{4cm}}
\hline
&\multicolumn{2}{c}{\textbf{Variables}}&&\multicolumn{2}{c}{\textbf{Rules}}\\
Nr  & Age  & hasjob     && \code{Age >= 0} & \code{IF Age < 15 THEN hasjob == `no'}\\
\cline{2-3}\cline{5-6}
1   & 36   & \code{yes} && 1        & \multicolumn{1}{c}{1}\\
2   & 53   & \code{NA}  && 1        & \multicolumn{1}{c}{\na{}}\\
3   & 11   & \code{yes} && 1        & \multicolumn{1}{c}{0}\\
\hline
\end{tabular}
\label{tab:example1}
\end{table}



In the example three records on age and work status are checked against two
rules: age must be larger than or equal to zero, and persons under 15 years old
cannot have a job. In each case the demand on  age can be checked and each
record passes this test, yielding $1$ (\waar{}) as validation result. For the
second rule, the first record passes the check since age equals 36 and the
person has a job which is allowed by the rule. In the second case the job
status is not available (\na{}). Hence the rule cannot be checked and the
returned value is \na{} as well. Finally, in the third record there is an
11-year old with a job which is a combination that is not allowed by the rule,
yielding $0$ (\onwaar{}).


The above discussion leads to the following definition of a validation result
%
\begin{definition}[Validation result]
A validation result is a value from the set $\{0,1,\na{}\}$.
\label{def:validationresult}
\end{definition}
Validation results are obtained as outcomes of evaluating a validation rule on
a data set. We also follow the common convention which identifies $0$ with
\onwaar{} and $1$ with \waar{}.  The numeric representation will make it easier
to define aggregators in later chapters. 


A recurring point of discussion is whether \na{} should be an allowed
validation result. The main other options are to either interrupt execution, or
to interpret the result as failed when one of the data points necessary for
evaluating a rule is missing. Because statistical data often suffers from
missing data the first option would yield many interruptions of a statistical
process flow. The only way to prevent that would be to make sure that each rule
guards against it explicitly by building in clauses that detect missing values.
The second option (\na{} implies \onwaar{}) would yield a loss of information.
More importantly, it assumes (possibly unwarranted) that the user of the result
follows this interpretation.

We deem both alternatives undesirable and therefore follow the definition
above, which also agrees with the definition of a formal data validation
function in the Methodology on Validation handbook published by the Validat
Foundation ESSnet \citep{zio2015methodology}.

Often, but not always, the dataset that is used to evaluate a validation rule
is also the data under scrutiny. As an example consider a set (column) of
numerical values $\la{x}=(x_1,x_2,\ldots,x_n)$. In a structural business survey
for example, $\la{x}$ may consist of profit values for different enterprises.
The rule 
\begin{align*}
\textsf{mean}(x) \geq 0
\end{align*}
is a check on the whole column. The dataset is considered invalid if the mean
profit is negative. This does not mean that all values are necessarily
erroneous, it just means that this particular combination of profit values
cannot be accepted. So in this case the data used to compute the result is also
the data that is being validated. Now consider a commonly used rule, based on a
method of \citet{hiridoglou1986statistical}, evaluated for each value $x_j$
($j=1,2,\ldots, n)$:
\begin{align}
\max\left(\frac{x_j}{x^*},\frac{x^*}{x_j}\right) \leq h.
\label{eq:hbfun}
\end{align}
Here, $x^*$ is a reference value, usually $\textsf{median}(x)$, and $h$ is a
fixed parameter. To evaluate this rule $x^*$ must be computed which implies
that the whole of $\la{x}$ must be known. This means that in a formal sense the
rule is a validation of $\la{x}$ as a whole. After all, changing any value
$x_{i}$ may influence the result for even when $i\not=j$. The aim is of course
to evaluate the rule $n$ times, once for each value $x_j$ where the goal is to
evaluate the value of $x_j$, not the combination of values $\la{x}$ $n$ times.

For this reason it is proposed that the validation report provides explicit
room to communicate cases where by intention, the validated data is not
identical to the data used in the evaluation of a rule. 
%
\begin{definition}[validation] 
A \emph{validation} is a tuple $(e,d,f,v)$, where $e$ identifies the physical
validation event, $d$ identifies the data points to which the result pertains,
$f$ identifies the evaluated validation rule, and $v$ is the generated
validation result.
\label{def:confrontation}
\end{definition}
%
We leave open the possibility that the components $e$, $f$ and $d$ consist of
informative tuples to identify subcomponents of the event, data, or rule.  In
particular, $d$ will contain information on the data used to evaluate the rule
as well as the data under scrutiny. Note that the above `definition' is not a
definition in the mathematical sense.  Rather it is a convention that can be
tested in a (software) environment where data, rules, and validation events
have been labeled and stored.






