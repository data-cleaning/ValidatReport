%%%%%%%%%%%%%%%%%%%%%%%%%%%%%%%%%%%%%%%%%%%%%%%%%%%%%%%%%%%%%%%%%%%%%%%%%%%%%%%
\section{Identifying validation results}
\label{sect:identifying}
To get an idea of the (meta)data necessary to identify a validation result,
consider the data, validation rules and validation results shown in
Table~\ref{tab:example1}. When a dataset is confronted with a validation rule,
there are three possible outcomes. A rule can be satisfied, yielding \waar{}, a
rule can be failed, yielding \onwaar{}, or a rule cannot be evaluated because
of missing data, yielding \na{}.
%
\begin{table}
\centering
\caption{Example data, validation rules, and results}
\begin{tabular}{rccccb{4cm}}
\hline
&\multicolumn{2}{c}{\textbf{Variables}}&&\multicolumn{2}{c}{\textbf{Rules}}\\
Nr  & Age  & hasjob     && \code{Age >= 0} & \code{IF Age < 15 THEN hasjob == `no'}\\
\cline{2-3}\cline{5-6}
1   & 36   & \code{yes} && \waar{}        & \multicolumn{1}{c}{\waar{}}\\
2   & 53   & \code{NA}  && \waar{}        & \multicolumn{1}{c}{\na{}}\\
3   & 11   & \code{yes} && \waar{}        & \multicolumn{1}{c}{\onwaar{}}\\
\hline
\end{tabular}
\label{tab:example1}
\end{table}



In the example three records on age and work status are checked against two
rules: age must be larger than or equal to zero, and persons under 15 years old
cannot have a job. In each case the demand on  age can be checked and and each
record passes this test, yielding \waar{} as validation result. For the second
rule, the first record passes the check since age equals 36 and the person has
a job which is allowed by the rule. In the second case the job status is not
available (\na{}). Hence the rule cannot be checked and the returned value is \na{}
as well. Finally, in the third record there is an 11-year old with a job which
is a combination that is not allowed by the rule, yielding \onwaar{}.


The above discussion leads to the following definition of a validation result.
%
\begin{definition}[Validation result]
A validation result is a value from the set $\{0,1,\na{}\}$.
\label{def:validationresult}
\end{definition}
Validation results are obtained as outcomes of evaluating a validation rule on
a data set. We also follow the common convention which identifies $0$ with
\onwaar{} and $1$ with \waar{}.  The numeric representation will make it easier
to define aggregators in later chapters. 


A recurring point of discussion is whether \na{} should be an allowed
validation result. The main other options are to either interrupt execution, or
to interpret the result as failed (\onwaar{}) when one of the data points
necessary for evaluating a rule is missing. Because statistical data often
suffers from missing data the first option would yield many interruptions of a
statistical process flow, unless each rule guards against it by building in
clauses that detect missing values explicitly. The second option (\na{} implies
\onwaar{}) would yield a loss of information. More importantly, it assumes
(possibly unwarranted) that the user of the result follows this interpretation.

We deem both alternatives undesirable and therefore follow the definition
above, which also agrees with the definition of a formal data validation
function in the Methodology on Validation handbook published by the Validat
Foundation \cite{zio2015methodology}.

\subsection{Validation events}

With a \emph{validation event} we mean the physical activity where a dataset is
confronted with a single rule, yielding precisely one validation result. A
validation event is characterized by its inputs (the rule and the dataset it
pertains to) as well as its own metadata, including the time of execution and
the actor performing the execution. We therefore adopt the following convention
by defining a set of keys that identifies such a physical event.
%
\begin{definition}[confrontation] 
A \emph{confrontation} is a triple $(t,r,d)$, where $t$ identifies the physical
validation event, $r$ identifies the evaluated validation rule and $d$
identifies the data points involved in the evaluation.
\label{def:validationevent}
\end{definition}
%
We leave open the possibility that the components $t$, $r$ and $d$ consist of
informative tuples to identify subcomponents of the event, data or rule. Note
that the above `definition' is not a definition in the mathematical sense.
Rather it is a convention that can be tested in a (software) environment where
data, rules, and validation events have been labeled and stored.


